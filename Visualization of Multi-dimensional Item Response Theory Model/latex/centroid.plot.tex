\HeaderA{centroid.plot}{Function to draw centroid plot}{centroid.plot}
\keyword{MIRT}{centroid.plot}
\keyword{Centroid}{centroid.plot}
\begin{Description}\relax
The function performs to draw a plot to show the relationship between the total score and the average of thetas of the examines who get the total score.
\end{Description}
\begin{Usage}
\begin{verbatim}
centroid.plot(a1, a2, d, theta1, theta2, score, correlation = 0, condition = TRUE, filename = "NA", type = "jpeg")
\end{verbatim}
\end{Usage}
\begin{Arguments}
\begin{ldescription}
\item[\code{a1}] The item discrimination parameters for theta1 
\item[\code{a2}] The item discrimination parameters for theta2 
\item[\code{d}] The item difficulty parameters 
\item[\code{theta1}] The theta 1 values of the examines
\item[\code{theta2}] The theta 2 values of the examines 
\item[\code{score}] The examines' total score 
\item[\code{correlation}] The correlation between theta 1 and theta 2. If the correlation does not equal to zero, the a1 and a2 will be transformed to maintain the orthogonal assumption of the two dimensions of thetas
\item[\code{condition}] A logical value to indicate if you need the function to plot the theta variance; if condition=TRUE, the conditional variance of the thetas will be shown in the plot, otherwise, it will not be shown. 
\item[\code{filename}] The filename in which the user saves the plot; if the filename is provided, the function will automatically save the plot for the users by the provided filename. 
\item[\code{type}] The format of files in which the user saves the plot 
\end{ldescription}
\end{Arguments}
\begin{Details}\relax
Four kinds of input methods are provided by this function: 1, a1, a2 and d. What the functions requires is item parameters and the examines' ability will be simulated;
2, a1, a2,d and theta. The function also allows the users to provide the item and examines' parameters at the same time; 3,theta and score. If the item parameters are not available, the users can only 
input the examines' theta and score; 4:score. The function even works if only score is available, however, the result returned by the function will not be accurate. Any of the four input methods will work well with the function.
\end{Details}
\begin{Value}
The x axis is the theta1; the y axis is theta2; the colored numbers represent the total score at the corresponding theta 1 and theta 2.
\end{Value}
\begin{Author}\relax
Zhan Shu, Terry Ackerman
\end{Author}
\begin{References}\relax
Ackerman,T.(1996),Graphical Representation of Multidimensional Item Response Theory Analyses,Applied Psychological Measurement,20(4),311-329.
\end{References}
\begin{Examples}
\begin{ExampleCode}
a1<-c(0.48,1.16,1.48,0.44,0.36,1.78,0.64,1.10,0.76,0.52,0.83,0.88,0.34,0.74,0.66)
a2<-c( 0.54,0.35,0.44,1.72,0.69,0.47,1.21,1.74,0.89,0.53,0.41,0.98,0.59,0.59,0.70)
d<-c( -1.11,0.29,1.51,-0.82,-1.89,-0.49,1.35,0.82,-0.21,-0.04,-0.68,0.22,-0.86,-1.33,1.21)
centroid.plot(a1,a2,d)
centroid.plot(a1,a2,d,condition=FALSE)
centroid.plot(a1,a2,d,filename="centroid",type="jepg")
\end{ExampleCode}
\end{Examples}

