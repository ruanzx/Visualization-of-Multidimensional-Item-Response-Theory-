\HeaderA{vector.plot}{Function to draw vectors to represent the items}{vector.plot}
\keyword{MIRT}{vector.plot}
\keyword{Vector}{vector.plot}
\begin{Description}\relax
The function performs to draw vectors to respresent the items.
\end{Description}
\begin{Usage}
\begin{verbatim}
vector.plot(a1, a2, d, correlation = 0, S_category = list(c(1:length(a1))), filename = "NA", type = "jpeg")
\end{verbatim}
\end{Usage}
\begin{Arguments}
\begin{ldescription}
\item[\code{a1}] The item parameters for theta 1
\item[\code{a2}] The item parameters for theta 2 
\item[\code{d}] The item difficulty parameters 
\item[\code{correlation}] The correlation between theta 1 and theta 2. If the correlation does not equal to zero, the a1 and a2 will be transformed to maintain the orthogonal assumption of the two dimensions of thetas 
\item[\code{S\_category}] The item categories  
\item[\code{filename}] The filename in which the user saves the plot; if the filename is provided, the function will automatically save the plot for the users by the provided filename.  
\item[\code{type}] The format of files in which the user saves the plot 
\end{ldescription}
\end{Arguments}
\begin{Details}\relax
The function allows the users to define the item groups and it will seperate the item groups by different color. The user has to use list to define the item categories, see the example.
\end{Details}
\begin{Author}\relax
Zhan Shu, Terry Ackerman
\end{Author}
\begin{References}\relax
Ackerman,T.(1996),Graphical Representation of Multidimensional Item Response Theory Analyses,Applied Psychological Measurement,20(4),311-329.
\end{References}
\begin{Examples}
\begin{ExampleCode}
a1<-c(0.48 , 1.16 , 1.48 , 0.44 , 0.36 , 1.78 , 0.64 , 1.10 , 0.76 , 0.52 , 0.83 ,0.88, 0.34 , 0.74 , 0.66)
a2<-c( 0.54, 0.35, 0.44, 1.72, 0.69, 0.47, 1.21, 1.74, 0.89, 0.53, 0.41, 0.98, 0.59, 0.59, 0.70)
d<-c( -1.11,0.29, 1.51,-0.82,-1.89,-0.49,1.35,0.82,-0.21,-0.04,-0.68, 0.22,-0.86,-1.33, 1.21)
vector.plot(a1,a2,d)
vector.plot(a1,a2,d,S_category=list(c(1,2,3),c(4,5,6,7,8,9,10,11,12,13,14,15))) # we define two groups of items: item 1,2 and 3 go into one group and the others form the other group.
\end{ExampleCode}
\end{Examples}

