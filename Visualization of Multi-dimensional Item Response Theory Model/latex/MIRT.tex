\HeaderA{MIRT}{Function to calculate the examines' positive response probability}{MIRT}
\keyword{MIRT}{MIRT}
\keyword{Model}{MIRT}
\begin{Description}\relax
The function performs to calculate postitive response probability based on the two dimensional compensatory IRT model.
\end{Description}
\begin{Usage}
\begin{verbatim}
MIRT(theta1, theta2, a1, a2, d)
\end{verbatim}
\end{Usage}
\begin{Arguments}
\begin{ldescription}
\item[\code{theta1}] The first dimension of examines' ability 
\item[\code{theta2}] The second dimension of examines' ability 
\item[\code{a1}] The item discrimation parameters of theta 1 
\item[\code{a2}] The item discrimation parameters of theta 2 
\item[\code{d}] The item difficulty parameters 
\end{ldescription}
\end{Arguments}
\begin{Value}
The function performs to provide a matrix of probability with items at the column and examines at the row.
\end{Value}
\begin{Author}\relax
Zhan Shu, Terry Ackerman
\end{Author}
\begin{References}\relax
Ackerman,T.(1996),Graphical Representation of Multidimensional Item Response Theory Analyses,Applied Psychological Measurement,20(4),311-329.\\
Reckase.M, McKinley.R,(1991), The Discriminating Power of Items That Measure More Than One Dimension, Applied Psychological Measurement,15(4),361-373.
\end{References}

