\HeaderA{contour.plot}{Function to draw the contour plot}{contour.plot}
\keyword{MIRT}{contour.plot}
\keyword{Contour}{contour.plot}
\begin{Description}\relax
The function is provide to draw the contour of the item or test surface.
\end{Description}
\begin{Usage}
\begin{verbatim}
contour.plot(a1, a2, d, correlation = 0, level = c(1:length(a1)), vector = FALSE, filename = "NA", type = "jpeg")
\end{verbatim}
\end{Usage}
\begin{Arguments}
\begin{ldescription}
\item[\code{a1}] The item discrimination parameters for theta 1 
\item[\code{a2}] The item discrimination parameters for theta 2 
\item[\code{d}] The item difficulty parameters 
\item[\code{correlation}] The correlation between theta 1 and theta 2. If the correlation does not equal to zero, a1 and a2 will be transformed to maintain the orthogonal assumption of the two dimensions of thetas 
\item[\code{level}] The levels of contour plot 
\item[\code{vector}] A logical value to indicate if you need to plot the item vector; if VECTOR=TRUE, the item vector will be shown in the plot, otherwise, it will not be shown. 
\item[\code{filename}] The filename in which the user saves the plot; if the filename is provided, the function will automatically save the plot for the users by the provided filename.  
\item[\code{type}] The format of files in which the user saves the plot 
\end{ldescription}
\end{Arguments}
\begin{Details}\relax
The function could plot the test/item contour, depending on the parameters you input. If only one sigle item parameter is provided, the function will plot item contour, otherwise, test contour will be plotted by 
the function.
\end{Details}
\begin{Author}\relax
Zhan Shu, Terry Ackerman
\end{Author}
\begin{References}\relax
Ackerman,T.(1996),Graphical Representation of Multidimensional Item Response Theory Analyses,Applied Psychological Measurement,20(4),311-329.
\end{References}
\begin{Examples}
\begin{ExampleCode}
a1<-c(0.48,1.16,1.48,0.44,0.36,1.78,0.64,1.10,0.76,0.52,0.83,0.88,0.34,0.74,0.66)
a2<-c(0.54, 0.35, 0.44, 1.72, 0.69, 0.47 1.21 1.74 0.89, 0.53, 0.41, 0.98, 0.59, 0.59, 0.70)
d<-c(-1.11,0.29, 1.51,-0.82,-1.89,-0.49,1.35,0.82,-0.21,-0.04,-0.68, 0.22,-0.86,-1.33, 1.21)
contour.plot(a1,a2,d) # test contour
contour.plot(a1[1],a2[1],d[1])# item contour
\end{ExampleCode}
\end{Examples}

