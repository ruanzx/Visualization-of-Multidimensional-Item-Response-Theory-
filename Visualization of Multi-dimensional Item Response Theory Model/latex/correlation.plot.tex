\HeaderA{correlation.plot}{Function to show the direction at which the composite thetas have the largest correlation with the total score}{correlation.plot}
\keyword{MIRT}{correlation.plot}
\keyword{correlation}{correlation.plot}
\begin{Description}\relax
The function will show at what angle the composite thetas will be maximumly correlated with the true score.
\end{Description}
\begin{Usage}
\begin{verbatim}
correlation.plot(a1, a2, d, theta1, theta2, score, correlation = 0, filename = "NA", type = "jpeg")
\end{verbatim}
\end{Usage}
\begin{Arguments}
\begin{ldescription}
\item[\code{a1}] The item discrimation parameter of theta 1 
\item[\code{a2}] The item discrimation parameter of theta 2
\item[\code{d}] The item difficulty parameter
\item[\code{theta1}] The examines' theta 1 
\item[\code{theta2}] The examines' theta 2 
\item[\code{score}] The examinees' score 
\item[\code{correlation}] The correlation between theta 1 and theta 2. If the correlation does not equal to zero, the a1 and a2 will be transformed to maintain the orthogonal assumption of the two dimensions of thetas 
\item[\code{filename}] The filename in which the user saves the plot; if the filename is provided, the function will automatically save the plot for the users by the provided filename. 
\item[\code{type}] The format of files in which the user saves the plot 
\end{ldescription}
\end{Arguments}
\begin{Details}\relax
The theta space in the plot is divided into eight sub-space based on the magnitude and signs of the two dimensions of thetas. The direction where the composite theta has the highest correlation with the total score is investigated within each sub-space. \\
The function allows four different combinations of the input parameters: 1, a1, a2 and d; 2, a1, a2, d and theta; 3, theta and score; 4, score. \\
The number within each sub-space is the direction (angle between the composite theta and theta 1), whose font size is related to the magnitude of the correlation. The greater the correlation, the greater the font size.
\end{Details}
\begin{Author}\relax
Zhan Shu, Terry Ackerman
\end{Author}
\begin{References}\relax
Ackerman,T.(1996),Graphical Representation of Multidimensional Item Response Theory Analyses,Applied Psychological Measurement,20(4),311-329.
\end{References}
\begin{Examples}
\begin{ExampleCode}
a1<-c(0.48,1.16 , 1.48 , 0.44 , 0.36 , 1.78 , 0.64 , 1.10 , 0.76 , 0.52 , 0.83 ,0.88, 0.34 , 0.74 , 0.66)
a2<-c( 0.54, 0.35, 0.44, 1.72, 0.69, 0.47 1.21 1.74 0.89, 0.53, 0.41, 0.98, 0.59, 0.59, 0.70)
d<-c( -1.11,0.29, 1.51,-0.82,-1.89,-0.49,1.35,0.82,-0.21,-0.04,-0.68, 0.22,-0.86,-1.33, 1.21)
correlation.plot(a1,a2,d)
\end{ExampleCode}
\end{Examples}

