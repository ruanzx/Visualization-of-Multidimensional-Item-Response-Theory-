\HeaderA{clamshell.plot}{Function to draw the clamshell plot}{clamshell.plot}
\keyword{MIRT}{clamshell.plot}
\keyword{clamshell}{clamshell.plot}
\begin{Description}\relax
The function is designed to draw the clamshell plot to show the magnitude of the test information at diferent theta positions and directions.
\end{Description}
\begin{Usage}
\begin{verbatim}
clamshell.plot(a1, a2, d, correlation=0, alpha =c(0, pi/18, pi/9, pi/6, pi/4.5, pi/3.6, pi/3, pi/2.57, pi/2.25, pi/2), scale = 0.01, filename = "NA", type = "jpeg")
\end{verbatim}
\end{Usage}
\begin{Arguments}
\begin{ldescription}
\item[\code{a1}] The item discrimation parameters for theta 1
\item[\code{a2}] The item discrimation parameters for theta 2
\item[\code{d}] The item difficulty parameters 
\item[\code{correlation}] The correlation between theta 1 and theta 2. If the correlation does not equal to zero, the a1 and a2 will be transformed to maintain the orthogonal assumption of the two dimensions of thetas 
\item[\code{alpha}] The information direction 
\item[\code{scale}] The control parameter of the plot size
\item[\code{filename}] The filename in which the user saves the plot; if the filename is provided, the function will automatically save the plot for the users by the provided filename.  
\item[\code{type}] The format of files in which the user saves the plot 
\end{ldescription}
\end{Arguments}
\begin{Details}\relax
The item/test information of MIRT has three key objects:1, the positive response probability; 2, the negative response probability; 3, the direction. 
In the two dimensional case, the information formula is P*(1-P)*(a1*cos(alpha1)+a2*cos(alpha2))^2,where alpha1+ alpha2=90. Therefore, both of the thetas and direction (alpha1 or alpha2) have effect on the test information.
The clamshell plot is to demonstrate how the information change along with the directions at the different theta coordinates.
\end{Details}
\begin{Author}\relax
Zhan Shu, Terry Ackerman
\end{Author}
\begin{References}\relax
Ackerman.T,(1994), Using Multidimensional Item Response Theory to understand what the items and tests are measuring, Applied Measurement in Education,7(4),255-278.\\
Ackerman.T, Girel.M,(2003), Using Multidimensional Item Response Theory to Evaluate Educational and Psychological Tests, Educational Measurement: Issues and Practices.
\end{References}
\begin{Examples}
\begin{ExampleCode}
a1<-c(0.48,1.16,1.48,0.44,0.36, 1.78 , 0.64 , 1.10 , 0.76 , 0.52 , 0.83 ,0.88, 0.34 , 0.74 , 0.66)
a2<-c( 0.54, 0.35, 0.44, 1.72, 0.69, 0.47 1.21 1.74 0.89, 0.53, 0.41, 0.98, 0.59, 0.59, 0.70)
d<-c( -1.11,0.29, 1.51,-0.82,-1.89,-0.49,1.35,0.82,-0.21,-0.04,-0.68, 0.22,-0.86,-1.33, 1.21)
clamshell.plot(a1,a2,d)
clamshell.plot(a1,a2,d,scale=0.1) # Change the size of clamshell plot
clamshell.plot(a1,a2,d,correlation=0.3) # check the impact of thetas' correlation on the information 
\end{ExampleCode}
\end{Examples}

