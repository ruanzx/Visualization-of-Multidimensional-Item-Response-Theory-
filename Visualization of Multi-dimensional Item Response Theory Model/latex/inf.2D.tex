\HeaderA{inf.2D}{Function to plot information surface in two-dimensional style}{inf.2D}
\keyword{MIRT}{inf.2D}
\keyword{inforamtion}{inf.2D}
\begin{Description}\relax
The function performs to provide the infromation surface at given information direction.
\end{Description}
\begin{Usage}
\begin{verbatim}
inf.2D(alpha, a1, a2, d, correlation = 0, azimuthal_angle = 0, colatitude_angle = 15, filename = "NA", type = "jpeg")
\end{verbatim}
\end{Usage}
\begin{Arguments}
\begin{ldescription}
\item[\code{alpha}] The direction of information (the angle between the theta 1 and the information direction) 
\item[\code{a1}] The item discrimination parameters for theta 1 
\item[\code{a2}] The item discrimination parameters for theta 2
\item[\code{d}] The item difficulty parameters
\item[\code{correlation}] The correlation between theta 1 and theta 2. If the correlation does not equal to zero, the a1 and a2 will be transformed to maintain the orthogonal assumption of the two dimensions of thetas 
\item[\code{azimuthal\_angle}] The azimuthal angle to view the plot  
\item[\code{colatitude\_angle}] The colatitude angle to view the plot 
\item[\code{filename}] The filename in which the user saves the plot; if the filename is provided, the function will automatically save the plot for the users by the provided filename. 
\item[\code{type}] The format of files in which the user saves the plot 
\end{ldescription}
\end{Arguments}
\begin{Details}\relax
The information is calculated based on the formula: inf=P*(1-P)*(a1*cos(alpha1)+a2*cos(alpha2))^2, where P is the positive response probability, 1-P is the negative response probability, a1 and a2 is the item discrimination 
parameters and alpha1+ alpha2=90.
\end{Details}
\begin{Author}\relax
Zhan Shu, Terry Ackerman
\end{Author}
\begin{References}\relax
Ackerman,T.(1996),Graphical Representation of Multidimensional Item Response Theory Analyses,Applied Psychological Measurement,20(4),311-329.\\
Reckase.M, McKinley.R,(1991), The Discriminating Power of Items That Measure More Than One Dimension, Applied Psychological Measurement,15(4),361-373.
\end{References}
\begin{Examples}
\begin{ExampleCode}
a1<-c(0.48 , 1.16 , 1.48 , 0.44 , 0.36 , 1.78 , 0.64 , 1.10 , 0.76 , 0.52 , 0.83 ,0.88, 0.34 , 0.74 , 0.66)
a2<-c( 0.54, 0.35, 0.44, 1.72, 0.69, 0.47, 1.21, 1.74, 0.89, 0.53, 0.41, 0.98, 0.59, 0.59, 0.70)
d<-c( -1.11,0.29, 1.51,-0.82,-1.89,-0.49,1.35,0.82,-0.21,-0.04,-0.68, 0.22,-0.86,-1.33, 1.21)
inf.2D(pi/3, a1, a2, d)
inf.2D(pi/2,a1,a2,d)# check the information at different direction
inf.2D(pi/2,a1,a2,d,azimuthal_angle=15) # change the view angle
inf.2D(pi/2,a1,a2,d,colatitude_angle=0) # change the view angle
\end{ExampleCode}
\end{Examples}

