\HeaderA{Angle}{Function to plot the angle}{Angle}
\keyword{MIRT}{Angle}
\keyword{Graph}{Angle}
\begin{Description}\relax
The function is provided to calculate the average of angles (information direction) where the test has maximum information for each examine with certain total score.
\end{Description}
\begin{Usage}
\begin{verbatim}
Angle(a1, a2, d, correlation = 0)
\end{verbatim}
\end{Usage}
\begin{Arguments}
\begin{ldescription}
\item[\code{a1}] The item discrimation parameters for theta 1 
\item[\code{a2}] The item discrimation parameters for theta 2
\item[\code{d}] The item diffiulcty parameters 
\item[\code{correlation}] The correlation between theta 1 and theta 2. If the correlation does not equal to zero, the a1 and a2 will be transformed to maintain the orthogonal assumption of the two dimensions of thetas 
\end{ldescription}
\end{Arguments}
\begin{Details}\relax
The function automatically simulates the two dimensions of thetas based on multivate normal distribution with mean zero and the provided correlation.
\end{Details}
\begin{Author}\relax
Zhan Shu, Terry Ackerman
\end{Author}
\begin{References}\relax
Ackerman,T.(1996),Graphical Representation of Multidimensional Item Response Theory Analyses,Applied Psychological Measurement,20(4),311-329.
\end{References}
\begin{Examples}
\begin{ExampleCode}
a1<-c(0.48,1.16,1.48,0.44,0.36,1.78,0.64,1.10,0.76,0.52,0.83,0.88,0.34,0.74,0.66)
a2<-c( 0.54,0.35,0.44,1.72,0.69,0.47,1.21,1.74,0.89,0.53,0.41,0.98,0.59,0.59,0.70)
d<-c( -1.11,0.29,1.51,-0.82,-1.89,-0.49,1.35,0.82,-0.21,-0.04,-0.68,0.22,-0.86,-1.33,1.21)
Angle(a1,a2,d,correlation=0)
Angle(a1,a2,d,correlation=0.3) # Checking the change of the angles.
\end{ExampleCode}
\end{Examples}

