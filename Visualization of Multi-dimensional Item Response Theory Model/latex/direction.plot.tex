\HeaderA{direction.plot}{Function to show the direction with maximum informatoin}{direction.plot}
\keyword{MIRT}{direction.plot}
\keyword{Direction}{direction.plot}
\begin{Description}\relax
The function investigates the direction (the angle between the information and the theta 1) at which the test/item has the maximum information for the given thetas.
\end{Description}
\begin{Usage}
\begin{verbatim}
direction.plot(a1, a2, d, correlation = 0, filename = "NA", type = "jpeg")
\end{verbatim}
\end{Usage}
\begin{Arguments}
\begin{ldescription}
\item[\code{a1}] The item discrimination parameters for theta 1 
\item[\code{a2}] The item discrimination parameters for theta 2 
\item[\code{d}] The item diffculty parameters 
\item[\code{correlation}] The correlation between theta 1 and theta 2. If the correlation does not equal to zero, the a1 and a2 will be transformed to maintain the orthogonal assumption of the two dimensions of thetas 
\item[\code{filename}] The filename in which the user saves the plot; if the filename is provided, the function will automatically save the plot for the users by the provided filename. 
\item[\code{type}] The format of files in which the user saves the plot  
\end{ldescription}
\end{Arguments}
\begin{Details}\relax
The angles, where the maximum information will be achieved at given theta positions,will be ploted in the graphic feactured with their font size related to the informaton magnitude (the greater the information, the greater the font size).
\end{Details}
\begin{Author}\relax
Zhan Shu, Terry Ackerman
\end{Author}
\begin{References}\relax
Ackerman,T.(1996),Graphical Representation of Multidimensional Item Response Theory Analyses,Applied Psychological Measurement,20(4),311-329.
\end{References}
\begin{Examples}
\begin{ExampleCode}
a1<-c(0.48 , 1.16 , 1.48 , 0.44 , 0.36 , 1.78 , 0.64 , 1.10 , 0.76 , 0.52 , 0.83 ,0.88, 0.34 , 0.74 , 0.66)
a2<-c( 0.54, 0.35, 0.44, 1.72, 0.69, 0.47, 1.21, 1.74, 0.89, 0.53, 0.41, 0.98, 0.59, 0.59, 0.70)
d<-c( -1.11,0.29, 1.51,-0.82,-1.89,-0.49,1.35,0.82,-0.21,-0.04,-0.68, 0.22,-0.86,-1.33, 1.21)
direction.plot(a1,a2,d, correlation=0.3)
\end{ExampleCode}
\end{Examples}

