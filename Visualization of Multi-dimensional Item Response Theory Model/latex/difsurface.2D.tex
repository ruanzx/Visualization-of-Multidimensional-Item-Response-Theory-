\HeaderA{difsurface.2D}{Function to plot two-dimensional difference surface}{difsurface.2D}
\keyword{MIRT}{difsurface.2D}
\keyword{Surface difference}{difsurface.2D}
\begin{Description}\relax
The function provides the difference surface between the two tests.
\end{Description}
\begin{Usage}
\begin{verbatim}
difsurface.2D(a1.1, a2.1, a1.2, a2.2, d1, d2, correlation = 0, azimuthal_angle = 0, colatitude_angle = 15, filename = "NA", type = "jpeg")
\end{verbatim}
\end{Usage}
\begin{Arguments}
\begin{ldescription}
\item[\code{a1.1}] The test 1's item discrimation parameters for theta 1  
\item[\code{a2.1}] The test 1's item discrimation parameters for theta 2 
\item[\code{a1.2}] The test 2's item discrimation parameters for theta 1 
\item[\code{a2.2}] The test 2's item discrimation parameters of theta 2 
\item[\code{d1}] The test 1's item difficulty parameters
\item[\code{d2}] The test 2's item difficulty parameters
\item[\code{correlation}] The correlation between theta 1 and theta 2. If the correlation does not equal to zero, the a1 and a2 will be transformed to maintain the orthogonal assumption of the two dimensions of thetas 
\item[\code{azimuthal\_angle}] The azimuthal viewing angle 
\item[\code{colatitude\_angle}] The colatitude viewing angle 
\item[\code{filename}] The filename in which the user saves the plot; if the filename is provided, the function will automatically save the plot for the users by the provided filename.  
\item[\code{type}] The format of files in which the user saves the plot 
\end{ldescription}
\end{Arguments}
\begin{Details}\relax
The function performs to plot a surface of the score difference between the two tests.
\end{Details}
\begin{Author}\relax
Zhan Shu, Terry Ackerman
\end{Author}
\begin{References}\relax
Ackerman,T.(1996),Graphical Representation of Multidimensional Item Response Theory Analyses,Applied Psychological Measurement,20(4),311-329.
\end{References}
\begin{Examples}
\begin{ExampleCode}
a1.1<-c(0.48 , 1.16 , 1.48 , 0.44 , 0.36 , 1.78 , 0.64 , 1.10 , 0.76 , 0.52 , 0.83 ,0.88, 0.34 , 0.74 , 0.66)
a2.1<-c( 0.54, 0.35, 0.44, 1.72, 0.69, 0.47, 1.21, 1.74, 0.89, 0.53, 0.41, 0.98, 0.59, 0.59, 0.70)
a1.2<-c(0.58 , 0.16 , 0.48 , 0.84 , 0.76 , 0.78 , 1.64 , 1.10 , 1.76 , 1.52 , 0.53 ,0.58, 0.54 , 0.84 , 0.76)
a2.2<-c( 1.54, 1.35, 1.44, 1.72, 1.69, 1.47, 0.21, 0.74, 0.99, 0.83, 0.71, 0.68, 0.79, 0.69, 0.78)
d1<-c( -1.11,0.29, 1.51,-0.82,-1.89,-0.49,1.35,0.82,-0.21,-0.04,-0.68, 0.22,-0.86,-1.33, 1.21)
d2<-c( -1.11,0.29, 1.51,-0.82,-1.89,-0.49,1.35,0.82,-0.21,-0.04,-0.68, 0.22,-0.86,-1.33, 1.21)
difsurface.2D(a1.1,a2.1,a1.2,a2.2,d1,d2,correlation=0.3)
\end{ExampleCode}
\end{Examples}

