\HeaderA{surface.3D}{Function to plot the item/test surface in three-dimensional style}{surface.3D}
\keyword{MIRT}{surface.3D}
\keyword{three-dimension surface}{surface.3D}
\begin{Description}\relax
The function performs to plot a three-dimensional item/test surface featured with free roataion.
\end{Description}
\begin{Usage}
\begin{verbatim}
surface.3D(a1, a2, d, correlation = 0)
\end{verbatim}
\end{Usage}
\begin{Arguments}
\begin{ldescription}
\item[\code{a1}] The item discrimination parameters of theta 1 
\item[\code{a2}] The item discriminatin parameters of theta 2 
\item[\code{d}] The item difficulty parameters 
\item[\code{correlation}] The correlation between theta 1 and theta 2. If the correlation does not equal to zero, the a1 and a2 will be transformed to maintain the orthogonal assumption of the two dimensions of thetas 
\end{ldescription}
\end{Arguments}
\begin{Details}\relax
The package is built based on "rgl" pacakge, and make sure it have been loaded before the use.
\end{Details}
\begin{Author}\relax
Zhan Shu, Terry Ackerman
\end{Author}
\begin{References}\relax
Ackerman,T.(1996),Graphical Representation of Multidimensional Item Response Theory Analyses,Applied Psychological Measurement,20(4),311-329.
\end{References}
\begin{Examples}
\begin{ExampleCode}
a1<-c(0.48 , 1.16 , 1.48 , 0.44 , 0.36 , 1.78 , 0.64 , 1.10 , 0.76 , 0.52 , 0.83 ,0.88, 0.34 , 0.74 , 0.66)
a2<-c( 0.54, 0.35, 0.44, 1.72, 0.69, 0.47, 1.21, 1.74, 0.89, 0.53, 0.41, 0.98, 0.59, 0.59, 0.70)
d<-c( -1.11,0.29, 1.51,-0.82,-1.89,-0.49,1.35,0.82,-0.21,-0.04,-0.68, 0.22,-0.86,-1.33, 1.21)
surface.3D(a1,a2,d,correlation=0.3)
\end{ExampleCode}
\end{Examples}

